\documentclass[runningheads,a4paper]{llncs}[2015/06/24]

%cmap has to be loaded before any font package (such as cfr-lm)
\usepackage{cmap}
\usepackage[T1]{fontenc}
\usepackage[references]{latex/agda}
\usepackage{catchfilebetweentags}

\usepackage{graphicx}

\usepackage{bbm}

\ifxetex
  \usepackage{fontspec}

  \setmonofont
    [ BoldFont       = DejaVuSansMono-Bold.ttf,
      ItalicFont     = DejaVuSansMono-Oblique.ttf,
      BoldItalicFont = DejaVuSansMono-BoldOblique.ttf,
      Scale          = MatchLowercase,
    ]
    {DejaVuSansMono.ttf}
\fi

% This handles the translation of unicode to latex:

% \usepackage{ucs}
\usepackage[utf8x]{inputenc}
\usepackage{autofe}


%Even though `american`, `english` and `USenglish` are synonyms for babel package (according to https://tex.stackexchange.com/questions/12775/babel-english-american-usenglish), the llncs document class is prepared to avoid the overriding of certain names (such as "Abstract." -> "Abstract" or "Fig." -> "Figure") when using `english`, but not when using the other 2.
%english has to go last to set it as default language
\usepackage[greek,ngerman,english]{babel}
%Hint by http://tex.stackexchange.com/a/321066/9075 -> enable "= as dashes
\addto\extrasenglish{\languageshorthands{ngerman}\useshorthands{"}}

%better font, similar to the default springer font
%cfr-lm is preferred over lmodern. Reasoning at http://tex.stackexchange.com/a/247543/9075
\usepackage[%
rm={oldstyle=false,proportional=true},%
sf={oldstyle=false,proportional=true},%
tt={oldstyle=false,proportional=true,variable=true},%
qt=false%
]{cfr-lm}
%
%if more space is needed, exchange cfr-lm by mathptmx
%\usepackage{mathptmx}

\usepackage{cite}
\usepackage{paralist}

%put figures inside a text
%\usepackage{picins}
%use
%\piccaptioninside
%\piccaption{...}
%\parpic[r]{\includegraphics ...}
%Text...

%for easy quotations: \enquote{text}
\usepackage{csquotes}

%enable margin kerning
\usepackage{microtype}

%tweak \url{...}
\usepackage{url}
%\urlstyle{same}
%improve wrapping of URLs - hint by http://tex.stackexchange.com/a/10419/9075
\makeatletter
\g@addto@macro{\UrlBreaks}{\UrlOrds}
\makeatother
%nicer // - solution by http://tex.stackexchange.com/a/98470/9075
%DO NOT ACTIVATE -> prevents line breaks
%\makeatletter
%\def\Url@twoslashes{\mathchar`\/\@ifnextchar/{\kern-.2em}{}}
%\g@addto@macro\UrlSpecials{\do\/{\Url@twoslashes}}
%\makeatother

%diagonal lines in a table - http://tex.stackexchange.com/questions/17745/diagonal-lines-in-table-cell
%slashbox is not available in texlive (due to licensing) and also gives bad results. This, we use diagbox
%\usepackage{diagbox}

%required for pdfcomment later
\usepackage{xcolor}


%enable nice comments
%this also loads hyperref
\usepackage{pdfcomment}
%enable hyperref without colors and without bookmarks
\hypersetup{hidelinks,
   colorlinks=true,
   allcolors=black,
   pdfstartview=Fit,
   breaklinks=true}
%enables correct jumping to figures when referencing
\usepackage[all]{hypcap}

\newcommand{\commentontext}[2]{\colorbox{yellow!60}{#1}\pdfcomment[color={0.234 0.867 0.211},hoffset=-6pt,voffset=10pt,opacity=0.5]{#2}}
\newcommand{\commentatside}[1]{\pdfcomment[color={0.045 0.278 0.643},icon=Note]{#1}}

%compatibality with packages todo, easy-todo, todonotes
\newcommand{\todo}[1]{\commentatside{#1}}
%compatiblity with package fixmetodonotes
\newcommand{\TODO}[1]{\commentatside{#1}}

%enable \cref{...} and \Cref{...} instead of \ref: Type of reference included in the link
\usepackage[capitalise,nameinlink]{cleveref}
%Nice formats for \cref
\crefname{section}{Sect.}{Sect.}
\Crefname{section}{Section}{Sections}

\usepackage{xspace}
%\newcommand{\eg}{e.\,g.\xspace}
%\newcommand{\ie}{i.\,e.\xspace}
\newcommand{\eg}{e.\,g.,\ }
\newcommand{\ie}{i.\,e.,\ }

%introduce \powerset - hint by http://matheplanet.com/matheplanet/nuke/html/viewtopic.php?topic=136492&post_id=997377
\DeclareFontFamily{U}{MnSymbolC}{}
\DeclareSymbolFont{MnSyC}{U}{MnSymbolC}{m}{n}
\DeclareFontShape{U}{MnSymbolC}{m}{n}{
    <-6>  MnSymbolC5
   <6-7>  MnSymbolC6
   <7-8>  MnSymbolC7
   <8-9>  MnSymbolC8
   <9-10> MnSymbolC9
  <10-12> MnSymbolC10
  <12->   MnSymbolC12%
}{}
\DeclareMathSymbol{\powerset}{\mathord}{MnSyC}{180}

% correct bad hyphenation here
\hyphenation{op-tical net-works semi-conduc-tor}

%% END COPYING HERE



\renewcommand\textPsi{\ensuremath{\Psi}}
\renewcommand\textXi{$\Xi$}
\renewcommand\textDelta{$\Delta$}
\renewcommand\textGamma{$\Gamma$}
\renewcommand\textsigma{$\sigma$}
\renewcommand\textSigma{$\Sigma$}
\renewcommand\textPi{$\Pi$}
\renewcommand\textrho{$\rho$}
\renewcommand\textphi{$\varphi$}
\renewcommand\textlambda{$\lambda$}
\renewcommand\texttau{$\tau$}
\renewcommand\texteta{$\eta$}
\renewcommand\textmu{$\mu$}
\renewcommand\textalpha{$\alpha$}
\renewcommand\textepsilon{$\epsilon$}
\renewcommand\textkappa{$\kappa$}
\renewcommand\textOmega{$\Omega$}
\renewcommand\textmho{$\mho$}
\renewcommand\textgamma{$\gamma$}
\renewcommand\textpi{$\varpi$}
\renewcommand\textbeta{\ensuremath{\beta}}
\renewcommand\textpsi{\ensuremath{\psi}}

\DeclareUnicodeCharacter{8994}{\ensuremath{\frown}}
\DeclareUnicodeCharacter{8759}{\ensuremath{::}}



\newcommand\SystemT{\textbf{System T}}
\newcommand\BI{\textbf{BI}}


\begin{document}

\title{Effectful Forcing and the Bar Principle}

\author{Jonathan Sterling }
\institute{Carnegie Mellon University}

\maketitle

\begin{abstract}
  We present in Agda a new proof of an old result: that the
  \SystemT{}-definable functionals of type 2 are closed under
  Brouwer's Bar Principle, a computational but non-constructive axiom
  which identifies two notions of well-founded approximation or cover
  in a space.
\end{abstract}

\section{Introduction}

In 2013, Mart\'in Escard\'o introduced a new technique in semantics
called \emph{effectful forcing}, a lightweight alternative to
sheaf-theoretic forcing which can be used to establish definability
and independence results~\cite{escardo:2013}. Effectful forcing is
remarkable in that it brings a battery of techniques known from
programming languages (for instance, monads, logical relations and
syntactic models) to bear on problems which have classically been
attacked using more traditional tools from mathematical logic, such as
sheaves and Beth-Kripke semantics.

Escard\'o used effectful forcing to establish that all
\SystemT{}-definable functionals of type 2 are continuous; in this
literate Agda pearl, we will use Escard\'o's method to show that the
class of \SystemT{}-definable functionals of type 2 are closed under
Brouwer's Bar Principle, a non-constructive but computationally
justified axiom, in a sense that we will make precise.

A more general version of this result is already well-known in the
literature~\cite{schwichtenberg:1979,oliva-steila:2016}; however, we
believe that our proof is of independent interest as an application of
effectful forcing.

\subsection{Brouwer's Bar Principle}

Brouwer's Bar Principle is usually stated in the form of an induction
schema (called ``Bar Induction'' or \BI{}); however, the real content
of the Bar Principle is most readily extracted from Brouwer's original
intent, which was to identify two distinct notions of approximation
(cover) in a space, through a transcendental argument in the style
typical of philosophical mathematics in the tradition of Kant and
Brouwer.

\subsubsection{Spreads, a constructive notion of space}

\ExecuteMetaData[latex/Spread/Core.tex]{spread}

\ExecuteMetaData[latex/Spread/Baire.tex]{spread}


\subsubsection*{Acknowledgments}
Thanks to Mark van Atten, Thierry Coquand, Mart\'in Escard\'o, Bob
Harper and Vincent Rahli for helpful discussions related to this
work. Thanks to Darin Morrison for his work on the
\texttt{agda-prelude} library which formed the basis for parts of this
development.




%%%%%%%%%%%%%%%%%%%%%%%%%%%%%%%%%%%%%%%%%%%%%%%%%%%%%%%%%%%%%%%%%%%%%%%%%%%%%%%
\nocite{van-atten:2004,brouwer:1981}
\bibliographystyle{splncs03}
\bibliography{references/refs}

%%%%%%%%%%%%%%%%%%%%%%%%%%%%%%%%%%%%%%%%%%%%%%%%%%%%%%%%%%%%%%%%%%%%%%%%%%%%%%%

\end{document}
